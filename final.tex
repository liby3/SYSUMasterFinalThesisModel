% !Mode:: "TeX:UTF-8"
\def\usewhat{pdflatex}                               % 定义编译方式 dvipdfmx 或者 pdflatex,默认为 dvipdfmx
                                                     % 方式编译,如果需要修改,只需改变花括号中的内容即可。
\documentclass[12pt,openright,a4paper,twoside]{book} %原本的设置
%\documentclass[12pt,openany,twoside]{book}           % 本科生毕业论文通常采用单页排版
% !Mode:: "TeX:UTF-8"
%  Authors: 张井   Jing Zhang: prayever@gmail.com     天津大学2010级管理与经济学部信息管理与信息系统专业硕士生
%           余蓝涛 Lantao Yu: lantaoyu1991@gmail.com  天津大学2008级精密仪器与光电子工程学院测控技术与仪器专业本科生

%%%%%%%%%% Package %%%%%%%%%%%%
\usepackage{graphicx}                       % 支持插图处理
\usepackage[a4paper,text={146.4true mm,239.2 true mm},top= 26.2true mm,left=31.8 true mm,head=6true mm,headsep=6.5true mm,foot=16.5true mm]{geometry}
                                            % 支持版面尺寸设置
\usepackage[squaren]{SIunits}               % 支持国际标准单位

\usepackage{titlesec}                       % 控制标题的宏包
\usepackage{titletoc}                       % 控制目录的宏包
\usepackage{fancyhdr}                       % fancyhdr宏包 支持页眉和页脚的相关定义
\usepackage[UTF8]{ctex}                     % 支持中文显示
\usepackage{CJKpunct}                       % 精细调整中文的标点符号
\usepackage{color}                          % 支持彩色
\usepackage{amsmath}                        % AMSLaTeX宏包 用来排出更加漂亮的公式
\usepackage{amssymb}                        % 数学符号生成命令
\usepackage[below]{placeins}    %允许上一个section的浮动图形出现在下一个section的开始部分,还提供\FloatBarrier命令,使所有未处理的浮动图形立即被处理
\usepackage{multirow}                       % 使用Multirow宏包,使得表格可以合并多个row格
\usepackage{booktabs}                       % 表格,横的粗线;\specialrule{1pt}{0pt}{0pt}
\usepackage{longtable}                      % 支持跨页的表格。
\usepackage{tabularx}                       % 自动设置表格的列宽
\usepackage{subfigure}                      % 支持子图 %centerlast 设置最后一行是否居中
\usepackage[subfigure]{ccaption}            % 支持子图的中文标题
\usepackage[sort&compress,numbers]{natbib}  % 支持引用缩写的宏包
\usepackage{enumitem}                       % 使用enumitem宏包,改变列表项的格式
\usepackage{calc}                           % 长度可以用+ - * / 进行计算
\usepackage{txfonts}                        % 字体宏包
\usepackage{bm}                             % 处理数学公式中的黑斜体的宏包
\usepackage[amsmath,thmmarks,hyperref]{ntheorem}  % 定理类环境宏包,其中 amsmath 选项用来兼容 AMS LaTeX 的宏包
\usepackage{CJKnumb}                        % 提供将阿拉伯数字转换成中文数字的命令
\usepackage{indentfirst}                    % 首行缩进宏包
\usepackage{CJKutf8}                        % 用在UTF8编码环境下,它可以自动调用CJK,同时针对UTF8编码作了设置
%\usepackage{hypbmsec}                      % 用来控制书签中标题显示内容
%\usepackage{enumerate}                      % 用于控制排序列表的编号样式
\newcommand{\tabincell}[2]{\begin{tabular}{@{}#1@{}}#2\end{tabular}}
\usepackage{xcolor}
%支持代码环境
\usepackage{listings}
\lstset{numbers=left,
language=[ANSI]{C},
numberstyle=\tiny,
extendedchars=false,
showstringspaces=false,
breakatwhitespace=false,
breaklines=true,
captionpos=b,
keywordstyle=\color{blue!70},
commentstyle=\color{red!50!green!50!blue!50},
frame=shadowbox,
rulesepcolor=\color{red!20!green!20!blue!20}
}
%支持算法环境
\usepackage[longend,linesnumbered,boxed,ruled,lined]{algorithm2e}
%\usepackage{algorithmic}

\usepackage{array}
\newcommand{\PreserveBackslash}[1]{\let\temp=\\#1\let\\=\temp}
\newcolumntype{C}[1]{>{\PreserveBackslash\centering}p{#1}}
\newcolumntype{R}[1]{>{\PreserveBackslash\raggedleft}p{#1}}
\newcolumntype{L}[1]{>{\PreserveBackslash\raggedright}p{#1}}

% 生成有书签的 pdf 及其生成方式。通常可以在 tjumain.tex 文件的第一行选择 pdflatex 或者是 dvipdfmx 编译手段。如果选择前者,则使用 pdflatex + pdflatex 编译; 如果选择后者,在编译的时候选择 latex + bibtex + latex + latex 编译。出现混淆的时候,系统会报错。
% 如果您的pdf制作中文书签有乱码使用如下命令,就可以解决了
\def\atemp{dvipdfmx}\ifx\atemp\usewhat
\usepackage[dvipdfmx,unicode,               % dvipdfmx 编译, 加入了中文复制,粘贴支持引擎。
            pdfstartview=FitH,
            bookmarksnumbered=true,
            bookmarksopen=true,
            colorlinks=false,
            pdfborder={0 0 1},
            citecolor=blue,
            linkcolor=black,
            anchorcolor=green,
            urlcolor=blue,
            breaklinks=true
            ]{hyperref}
\fi

\def\atemp{pdflatex}\ifx\atemp\usewhat
\usepackage{cmap}                           % pdflatex 编译时,可以生成可复制、粘贴的中文 PDF 文档, 缺点是在Windows上显示时效果不大好,字体发虚
\usepackage[pdftex,unicode,
            %CJKbookmarks=true,
            bookmarksnumbered=true,
            bookmarksopen=true,
            colorlinks=false,
            pdfborder={0 0 0},%pdfborder={0 0 1}就能显示红色框
            citecolor=blue,
            linkcolor=red,
            anchorcolor=green,
            urlcolor=blue,
            breaklinks=true
            ]{hyperref}
\fi

\usepackage{setspace}   % 用于目录设置行距


\usepackage{caption}

                                % 定义本文所使用宏包
\usepackage{graphicx}
\usepackage{subfigure}
\usepackage{epstopdf}                                 %支持eps图片转换成PDF格式
\usepackage{tipa}
\graphicspath{{figures/}}                            % 定义所有的 .eps 文件在 figures 子目录下
\begin{document}                                     % 开始全文
\begin{CJK*}{UTF8}{song}                             % 开始中文字体使用
% !Mode:: "TeX:UTF-8"

%%%%%%%%%%%%%%%%% Fonts Definition and Basics %%%%%%%%%%%%%%%%%
\newcommand{\song}{\CJKfamily{song}}    % 宋体
\newcommand{\fs}{\CJKfamily{fs}}        % 仿宋体
\newcommand{\kai}{\CJKfamily{kai}}      % 楷体
\newcommand{\hei}{\CJKfamily{hei}}      % 黑体
\newcommand{\li}{\CJKfamily{li}}        % 隶书
\newcommand{\chuhao}{\fontsize{28pt}{28pt}\selectfont}       % 初号, 单倍行距
\newcommand{\yihao}{\fontsize{26pt}{26pt}\selectfont}       % 一号, 单倍行距
\newcommand{\xiaoyi}{\fontsize{24pt}{24pt}\selectfont}      % 小一, 单倍行距
\newcommand{\erhao}{\fontsize{22pt}{1.25\baselineskip}\selectfont}       % 二号, 1.25 倍行距
\newcommand{\xiaoer}{\fontsize{18pt}{18pt}\selectfont}      % 小二, 单倍行距
\newcommand{\sanhao}{\fontsize{16pt}{16pt}\selectfont}      % 三号, 单倍行距
\newcommand{\xiaosan}{\fontsize{15pt}{15pt}\selectfont}     % 小三, 单倍行距
\newcommand{\sihao}{\fontsize{14pt}{14pt}\selectfont}       % 四号, 单倍行距
\newcommand{\xiaosi}{\fontsize{12pt}{12pt}\selectfont}      % 小四, 单倍行距
\newcommand{\wuhao}{\fontsize{10.5pt}{10.5pt}\selectfont}   % 五号, 单倍行距
\newcommand{\xiaowu}{\fontsize{9pt}{9pt}\selectfont}        % 小五, 单倍行距

\CJKtilde  % 重新定义了波浪符~的意义
\newcommand\prechaptername{第}
\newcommand\postchaptername{章}

\punctstyle{hangmobanjiao}             % 调整中文字符的表示,行内占一个字符宽度,行尾占半个字符宽度

% 调整罗列环境的布局
\setitemize{leftmargin=3em,itemsep=0em,partopsep=0em,parsep=0em,topsep=-0em}
\setenumerate{leftmargin=3em,itemsep=0em,partopsep=0em,parsep=0em,topsep=0em}

% 避免宏包 hyperref 和 arydshln 不兼容带来的目录链接失效的问题。
\def\temp{\relax}
\let\temp\addcontentsline
\gdef\addcontentsline{\phantomsection\temp}

% 自定义项目列表标签及格式 \begin{publist} 列表项 \end{publist}
\newcounter{pubctr} %自定义新计数器
\newenvironment{publist}{%%%%%定义新环境
\begin{list}{[\arabic{pubctr}]} %%标签格式
    {
     \usecounter{pubctr}
     \setlength{\leftmargin}{2.5em}   % 左边界 \leftmargin =\itemindent + \labelwidth + \labelsep
     \setlength{\itemindent}{0em}     % 标号缩进量
     \setlength{\labelsep}{1em}       % 标号和列表项之间的距离,默认0.5em
     \setlength{\rightmargin}{0em}    % 右边界
     \setlength{\topsep}{0ex}         % 列表到上下文的垂直距离
     \setlength{\parsep}{0ex}         % 段落间距
     \setlength{\itemsep}{0ex}        % 标签间距
     \setlength{\listparindent}{0pt}  % 段落缩进量
    }}
{\end{list}}

\makeatletter
\renewcommand\normalsize{
  \@setfontsize\normalsize{12pt}{12pt} % 小四对应 12 pt
  \setlength\abovedisplayskip{4pt}
  \setlength\abovedisplayshortskip{4pt}
  \setlength\belowdisplayskip{\abovedisplayskip}
  \setlength\belowdisplayshortskip{\abovedisplayshortskip}
  \let\@listi\@listI}
\def\defaultfont{\renewcommand{\baselinestretch}{2.0}\normalsize\selectfont} % 设置行距

\renewcommand{\CJKglue}{\hskip -0.1 pt plus 0.08\baselineskip} % 控制字间距,使每行 34 个汉字
\makeatother

%%%%%%%%%%%%% Contents %%%%%%%%%%%%%%%%%
\renewcommand{\contentsname}{目\qquad 录}
\setcounter{tocdepth}{1} % 控制目录深度   //只显示两级目录
\titlecontents{chapter}[2em]{\vspace{.5\baselineskip}\xiaosi\hei}
             %{\prechaptername\CJKnumber{\thecontentslabel}\postchaptername\qquad}{}   % 让目录章节使用“第一章”
             {\prechaptername~\thecontentslabel~\postchaptername\quad}{}               % 让目录章节使用“第1章”
             {\hspace{.5em}\titlerule*[6pt]{\textbf{$\cdot$}}\xiaosi\contentspage}
%\titlecontents{section}[3.8em]{\vspace{.25\baselineskip}\xiaosi\song}
%             {\thecontentslabel\quad}{}
%             {\hspace{.5em}\titlerule*[6pt]{$\cdot$}\xiaosi\contentspage}
\titlecontents{section}[2em]{\vspace{.25\baselineskip}\xiaosi\song}
			  {\thecontentslabel\quad}{}
			  {\hspace{.5em}\titlerule*[6pt]{$\cdot$}\xiaosi\contentspage}
\titlecontents{subsection}[6.1em]{\vspace{.25\baselineskip}\xiaosi\song}
             {\thecontentslabel\quad}{}
             {\hspace{.5em}\titlerule*[6pt]{$\cdot$}\xiaosi\contentspage}

%%%%%%%%%% Chapter and Section %%%%%%%%%%%%%
\setcounter{secnumdepth}{2}     %使得当前章节编号深度增加或减小,num可取正值或负值
\setlength{\parindent}{2em}
\renewcommand{\chaptername}{\prechaptername{~\thechapter~}\postchaptername}     % 阿拉伯数字转换为中文数字:CJKnumber{\thechapter}

\titleformat{\chapter}{\centering\xiaoer\hei}{\hei\chaptername}{2em}{}
\titlespacing{\chapter}{0pt}{0.1\baselineskip}{0.8\baselineskip}
\titleformat{\section}{\xiaosan\song\bfseries}{\thesection}{1em}{}
\titlespacing{\section}{0pt}{0.15\baselineskip}{0.25\baselineskip}
\titleformat{\subsection}{\sihao\song}{\thesubsection}{1em}{}
\titlespacing{\subsection}{0pt}{0.1\baselineskip}{0.3\baselineskip}
\titleformat{\subsubsection}{\sihao\hei}{\thesubsubsection}{1em}{}
\titlespacing{\subsubsection}{0pt}{0.05\baselineskip}{0.1\baselineskip}

%======================= 定义列表项目格式 ==========================%
%\renewcommand\labelenumi{\textcircled{\scriptsize \theenumi}}  %带圈的数字
\renewcommand\labelenumi{(\theenumi)}   % 带括号的数字,如(1)
\renewcommand\labelenumii{(\theenumii)}
\renewcommand\labelenumiii{\theenumiii.}
\renewcommand\labelenumiv{\theenumiv.}

%%%%%%%%%% Table, Figure and Equation %%%%%%%%%%%%%%%%%
\renewcommand{\tablename}{表}                                     % 插表题头
\renewcommand{\figurename}{图}                                    % 插图题头
\renewcommand{\thefigure}{\arabic{chapter}-\arabic{figure}}       % 使图编号为 7-1 的格式 %\protect{~}


\renewcommand{\thesubfigure}{(\alph{subfigure})}                   % 使子图编号为 a) 的格式

\renewcommand{\thesubtable}{(\alph{subtable})}                    % 使子表编号为 (a) 的格式
\renewcommand{\thetable}{\arabic{chapter}-\arabic{table}}         % 使表编号为 7-1 的格式
\renewcommand{\theequation}{\arabic{chapter}-\arabic{equation}}   % 使公式编号为 7-1 的格式

%------------------------- 列表与图表距离设置 -----------------------%
\setlength{\topsep}{3pt plus1pt minus2pt}           % 第一个item和前面版落间的距离
\setlength{\partopsep}{3pt plus1pt minus2pt}        % 当在一个新页开始时加到 % \topsep的额外空间
\setlength{\itemsep}{3pt plus 1pt minus2pt}          % 连续items之间的距离.
\setlength{\floatsep}{10pt plus 3pt minus 2pt}      % 图形之间或图形与正文之间的距离
\setlength{\abovecaptionskip}{6pt plus.1pt minus.1pt} % 图形中的图与标题之间的距离
\setlength{\belowcaptionskip}{6pt plus.1pt minus.1pt} % 表格中的表与标题之间的距离

%下面这组命令使浮动对象的缺省值稍微宽松一点,从而防止幅度
%对象占据过多的文本页面,也可以防止在很大空白的浮动页上放置
%很小的图形。
\renewcommand{\textfraction}{0.15}
\renewcommand{\topfraction}{0.85}
\renewcommand{\bottomfraction}{0.3}
\renewcommand{\floatpagefraction}{0.90}

%%%%%% 定制浮动图形和表格标题样式 %%%%%%
\makeatletter
\long\def\@makecaption#1#2{
   \vskip\abovecaptionskip
   \sbox\@tempboxa{\centering\wuhao\song{#1\quad #2} }      %修改图编号和标题间的间距
   \ifdim \wd\@tempboxa >\hsize
     \centering\wuhao\song{#1\quad #2} \par                  %修改表编号和标题间的间距
   \else
     \global \@minipagefalse
     \hb@xt@\hsize{\hfil\box\@tempboxa\hfil}
   \fi
   \vskip\belowcaptionskip}
\makeatother
\captiondelim{~~~~} %用来控制longtable表头分隔符

%%%%%%%%%% Theorem Environment %%%%%%%%%%%%%%%%%
\theoremstyle{plain}
\theorembodyfont{\song\rmfamily}
\theoremheaderfont{\hei\rmfamily}
\newtheorem{defi}{定义~}[chapter]
\newtheorem{theorem}{定理~}[chapter]
\newtheorem{lemma}{引理~}[chapter]
\newtheorem{axiom}{公理~}[chapter]
\newtheorem{proposition}{命题~}[chapter]
\newtheorem{prop}{性质~}[chapter]
\newtheorem{corollary}{推论~}[chapter]
\newtheorem{conclusion}{结论~}[chapter]
\newtheorem{definition}{定义~}[chapter]
\newtheorem{conjecture}{猜想~}[chapter]
\newtheorem{example}{例~}[chapter]
\newtheorem{remark}{注~}[chapter]
%\newtheorem{algorithm}{算法~}[chapter]
\newenvironment{proof}{\noindent{\hei 证明:}}{\hfill $ \square $ \vskip 4mm}
\theoremsymbol{$\square$}

%%%%%%%%%% Page: number, header and footer  %%%%%%%%%%%%%%%%%

%\frontmatter 或 \pagenumbering{roman}
%\mainmatter 或 \pagenumbering{arabic}
\makeatletter
\renewcommand\frontmatter{\clearpage
  \@mainmatterfalse
  }
\makeatother

%%%%%%%%%%%% References %%%%%%%%%%%%%%%%%
\renewcommand{\bibname}{参考文献}
% 重定义参考文献样式,来自thu
\makeatletter
\renewenvironment{thebibliography}[1]{
    \titleformat{\chapter}{\center\sihao\hei}{\chaptername}{2em}{}
   \chapter*{\bibname}
   \wuhao
   \list{\@biblabel{\@arabic\c@enumiv}}
        {\renewcommand{\makelabel}[1]{##1\hfill}
         \settowidth\labelwidth{0 cm}
         \setlength{\labelsep}{0pt}
         \setlength{\itemindent}{0pt}
         \setlength{\leftmargin}{\labelwidth+\labelsep}
         \addtolength{\itemsep}{-0.7em}
         \usecounter{enumiv}
         \let\p@enumiv\@empty
         \renewcommand\theenumiv{\@arabic\c@enumiv}}
    \sloppy\frenchspacing
    \clubpenalty4000
    \@clubpenalty \clubpenalty
    \widowpenalty4000
    \interlinepenalty4000
    \sfcode`\.\@m}
   {\def\@noitemerr
     {\@latex@warning{Empty `thebibliography' environment}}
    \endlist\frenchspacing}
\makeatother

\addtolength{\bibsep}{-0.5em}     % 缩小参考文献间的垂直间距
\setlength{\bibhang}{2em}         % 每个条目自第二行起缩进的距离



% 参考文献引用作为上标出现
%\newcommand{\citeup}[1]{\textsuperscript{\cite{#1}}}
\makeatletter
    \def\@cite#1#2{\textsuperscript{[{#1\if@tempswa , #2\fi}]}}
\makeatother
%% 引用格式
\bibpunct{[}{]}{,}{s}{}{,}

%%%%%%%%%%%% Cover %%%%%%%%%%%%%%%%%
% 封面、摘要、版权、致谢格式定义
\makeatletter
\def\ctitle#1{\def\@ctitle{#1}}\def\@ctitle{}
\def\etitle#1{\def\@etitle{#1}}\def\@etitle{}
\def\etitlefirstline#1{\def\@etitlefirstline{#1}}\def\@etitlefirstline{}
\def\etitlesecondline#1{\def\@etitlesecondline{#1}}\def\@etitlesecondline{}
\def\csubject#1{\def\@csubject{#1}}\def\@csubject{}
\def\esubject#1{\def\@esubject{#1}}\def\@esubject{}
\def\cauthor#1{\def\@cauthor{#1}}\def\@cauthor{}
\def\eauthor#1{\def\@eauthor{#1}}\def\@eauthor{}
\def\csupervisor#1{\def\@csupervisor{#1}}\def\@csupervisor{}
\def\esupervisor#1{\def\@esupervisor{#1}}\def\@esupervisor{}
\def\cdate#1{\def\@cdate{#1}}\def\@cdate{}
\long\def\cabstract#1{\long\def\@cabstract{#1}}\long\def\@cabstract{}
\long\def\eabstract#1{\long\def\@eabstract{#1}}\long\def\@eabstract{}
\def\ckeywords#1{\def\@ckeywords{#1}}\def\@ckeywords{}
\def\ekeywords#1{\def\@ekeywords{#1}}\def\@ekeywords{}
\def\cheading#1{\def\@cheading{#1}}\def\@cheading{}


\pagestyle{fancy}
    \renewcommand{\chaptermark}[1]%
    {\markboth{\chaptername \ #1}{}}            % \chaptermark 去掉章节标题中的数字
    \renewcommand{\sectionmark}[1]%
    {\markright{\thesection \ #1}{}}            % \sectionmark 去掉章节标题中的数字
    \fancyhf{}
    %\fancyhead{\song\wuhao \@ctitle}  % 页眉
    %\lhead{\song\wuhao \@ctitle}      % 左页眉,   \rightmark 在 article 中包含 subsection 信息,在 report 和 book 中包含 section 信息
    \lhead{\song\wuhao {XXXX}}      % 左页眉
    \rhead{\song\wuhao \leftmark}    % 右页眉,\leftmark 在 article 中包含section的信息,在 report 和 book 中包含 chapter 的信息
    \fancyfoot[C]{\song\xiaowu -~\thepage~-}
\newlength{\@title@width}


% 定义封面
\def\makecover{
%\cleardoublepage%
   \phantomsection
    \pdfbookmark[-1]{\@ctitle}{ctitle}

\begin{titlepage}
\vspace*{10pt}
\begin{center}

  \vspace*{10pt}
  \hei\chuhao{\textbf{中山大学硕士学位论文}}

  \vspace*{60pt}
  \song\xiaoer\textbf{\@ctitle}

  \xiaoer{\textbf{\@etitle}}

  \begin{spacing}{2.0}
  \vspace*{42pt}
  \setlength{\@title@width}{6cm}
  {\sihao\song{{

  \begin{tabular}{lc}
    学~~位~~申~~请~~人:   &  \underline{\makebox[\@title@width][c]{\@cauthor}}\\
    %\underline{\makebox[\@title@width][c]{~}} \\ 
    导师姓名及职称:       &  \underline{\makebox[\@title@width][c]{\@csupervisor}} \\
    %\underline{\makebox[\@title@width][c]{~}} \\
    专~~~~业~~~~名~~~~称: &  \underline{\makebox[\@title@width][c]{\@csubject}}\\
  \end{tabular}}}
 }

 \end{spacing}

  \vspace*{10pt}

 \begin{spacing}{2.0}
 \vspace*{42pt}
  \setlength{\@title@width}{5cm}
  {\sanhao\song{{
  \begin{tabular}{lc}
    答辩委员会主席(签名):  &  \underline{\makebox[\@title@width][c]{~}} \\
    答辩委员会委员(签名):  &  \underline{\makebox[\@title@width][c]{~}} \\
    ~ &  \underline{\makebox[\@title@width][c]{~}}\\
    ~ &  \underline{\makebox[\@title@width][c]{~}}\\
    ~ &  \underline{\makebox[\@title@width][c]{~}}\\
    ~ &  \underline{\makebox[\@title@width][c]{~}}\\
  \end{tabular}}}
 }
 \end{spacing}
 \vspace*{32pt}

\song\sihao{{XXXX~ 年~ XX~ 月}}
\end{center}
\end{titlepage}

% 空白页
\newpage
\thispagestyle{empty}
\mbox{}


%%%%%%%%%%%%%%%%%%%   Originality Statement  %%%%%%%%%%%%%%%%%%%%%%%
\clearpage
\pdfbookmark[0]{论文原创性声明}{originality}
\chapter*{\centering\sanhao\song\bfseries 论文原创性声明}
\song\defaultfont
本人郑重声明:所呈交的学位论文,是本人在导师的指导下,独立进行研究工作所取得的成果。除文中已经注明引用的内容外,本论文不包含任何其他个人或集体已经发表或撰写过的作品成果。对本文的研究作出重要贡献的个人和集体,均已在文中以明确方式标明。本人完全意识到本声明的法律结果由本人承担。

\vspace*{40pt}
\begin{flushright}
\setlength{\@title@width}{5cm}
  {\sihao\song{
  \begin{tabular}{lc}
    学位论文作者签名:           &  \underline{\makebox[\@title@width][c]{~}} \\
    \qquad\qquad\qquad 日~~期:  &  \underline{\makebox[\@title@width][c]{~}} \\
  \end{tabular}}
 }
\end{flushright}

%%%%%%%%%%%%%%%%%%%   Authorization Statement  %%%%%%%%%%%%%%%%%%%%%%%
\vspace*{60pt}
\pdfbookmark[0]{学位论文使用授权声明}{authorization}
\begin{center}
  \sanhao\song\bfseries{学位论文使用授权声明}
\end{center}

\song\defaultfont
本人完全了解中山大学有关保留、使用学位论文的规定,即:学校有权保留学位论文并向国家主管部门或其指定机构送交论文的电子版和纸质版,有权将学位论文用于非赢利目的的少量复制并允许论文进入学校图书馆、院系资料室被查阅,有权将学位论文的内容编入有关数据库进行检索,可以采用复印、缩印或其他方法保存学位论文。

\vspace*{40pt}
\begin{center}
\setlength{\@title@width}{5cm}
  {\sihao\song{
  \begin{tabular}{ll}
    学位论文作者签名: \qquad\qquad\qquad\qquad\qquad  &  导师签名: \qquad\qquad\qquad\\
    日期: \qquad 年\qquad 月\qquad 日     &  日期: \qquad 年\qquad 月\qquad 日 \\
  \end{tabular}}
 }
\end{center}
\thispagestyle{empty}   % 用于设置 论文原创性声明页 无页眉页脚

% --------------------空白页,方便打印双页-------------------
\newpage
\thispagestyle{empty}   % % 用于设置 空白页 无页眉页脚
\mbox{}


%%%%%%%%%%%%%%%%%%%   Abstract and Keywords  %%%%%%%%%%%%%%%%%%%%%%%
\clearpage
\setcounter{page}{1}                    % 重新开始页码
\pagenumbering{roman}                   %罗马数字页码

\markboth{摘~要}{摘~要}
\pdfbookmark[0]{摘~~要}{cabstract}
%\newpage

\begin{flushleft}
\setlength{\@title@width}{5cm}
  {\wuhao\song{
  \begin{tabular}{ll}
    论文题目:      &  {XXXXX }\\
    专~~~~~~~~业:  &  \@csubject \\
    硕~~士~~生:    &  \@cauthor \\
    指导教师:      &  \@csupervisor \\
  \end{tabular}}
 }
\end{flushleft}

%\addcontentsline{toc}{chapter}{摘~要}
\vspace{\baselineskip}
\begin{center}
\sanhao\song\bfseries 摘\qquad 要
\end{center}

\song\defaultfont
\@cabstract
\vspace{\baselineskip}

\hangafter=1\hangindent=52.3pt\noindent
{\hei\xiaosi 关键词:} \@ckeywords
%\thispagestyle{empty}

% --------------------空白页,方便打印双页-------------------
\newpage
\thispagestyle{empty}   % % 用于设置 空白页 无页眉页脚
\mbox{}

%%%%%%%%%%%%%%%%%%%   English Abstract  %%%%%%%%%%%%%%%%%%%%%%%%%%%%%%
\clearpage
%\phantomsection
\fancypagestyle{plain}{
    \fancyhf{}
    \lhead{\xiaowu {XXXXXXXXXXXXXXXX}}
    %\lhead{\xiaowu \@etitle}                        % 左页眉
    \rhead{\xiaowu \leftmark}                       % 右页眉
    \fancyfoot[C]{\song\xiaowu -~\thepage~-}           % 首页页脚格式
}
\thispagestyle{plain}
\markboth{ABSTRACT}{ABSTRACT}
\pdfbookmark[0]{ABSTRACT}{eabstract}
%\newpage

\begin{flushleft}
\setlength{\@title@width}{5cm}
  {\wuhao{
  \begin{tabular}{ll}
    Title:  & \@etitlefirstline \\& \@etitlesecondline\\
    Major:      &  \@esubject \\
    Name:       &  \@eauthor \\
    Supervisor: &  \@esupervisor \\
  \end{tabular}}
}
\end{flushleft}


\vspace{\baselineskip}
\begin{center}
\sanhao{\bf{Abstract}}
\end{center}

%\vspace{\baselineskip}
\@eabstract
\vspace{\baselineskip}

\hangafter=1\hangindent=60pt\noindent
\textbf{Keywords: } \@ekeywords
\thispagestyle{plain}   % 用于设置多于一页的英文摘要页眉
}
\makeatother
                                 % 完成对论文各个部分格式的设置

\frontmatter                                         % 以下是论文导言部分,包括论文的封面,中英文摘要和中文目录
%\fancypagestyle{plain}{
%\fancyhf{}
%\lhead{\song\wuhao \@ctitle}  % 左页眉
%\rhead{\song\wuhao \leftmark}    % 右页眉
%\fancyfoot[C]{\song\xiaowu~\thepage~}
%\renewcommand{\headrulewidth}{0 pt}
%}

%%%%%%%%%%   封面   %%%%%%%%%%
% !Mode:: "TeX:UTF-8"

%%  可通过增加或减少 setup/format.tex中的
%%  第274行 \setlength{\@title@width}{8cm}中 8cm 这个参数来 控制封面中下划线的长度。
\cheading{中山大学硕士学位论文}      % 设置正文的页眉,需要填上对应的毕业年份
\ctitle{AAA}    % 封面用论文标题,自己可手动断行
\etitle{AAA}    %论文英文标题
\csubject{AAA}   % 专业名称
\esubject{AAA}
\cauthor{AAA}            % 学生姓名
\eauthor{AAA}
\csupervisor{AAA}        % 导师姓名
\esupervisor{AAA}
\etitlefirstline{AAA}
\etitlesecondline{AAA}
%\cdate{\the\year~\the\month~月~\the\day~日}
\cdate{AAAA~年~A~月~AA~日}

\cabstract{
	XXXXXX
}
\ckeywords{XXX, XX, X, XXXXXXX}

\eabstract{
XXXXX
}

\ekeywords{XXX, XX, X, XXXXXXX}

\makecover

\clearpage

                                % 封面
\renewcommand{\algorithmcfname}{算法}
%%%%%%%%%%   目录   %%%%%%%%%%
\defaultfont
\clearpage{\pagestyle{empty}\cleardoublepage}       %本有
%\setcounter{page}{1}                                % 单独从 1 开始编页码
%\pagenumbering{Roman}
\titleformat{\chapter}{\centering\sanhao\hei}{\chaptername}{2em}{} % 设置目录两字的格式
\pdfbookmark[0]{目~~录}{mulu}
\fancypagestyle{plain}{
	\fancyhf{}
	\fancyhead[RO]{\song\xiaosi \leftmark}
	\fancyhead[LE]{\song\xiaosi \@ctitle}
	%\lhead{\song\wuhao \@ctitle}  % 左页眉
	%\rhead{\song\wuhao \leftmark}    % 右页眉
	%\renewcommand{\headrulewidth}{0 pt}
	\renewcommand{\headrulewidth}{0.5pt}
	\renewcommand{\footrulewidth}{0pt}
	\fancyfoot[C]{\song\xiaowu~\thepage~}
}
\begin{spacing}{1.2}
\tableofcontents                                     % 中文目录
\end{spacing}
\thispagestyle{fancy}
%\newpage
%\thispagestyle{empty}
%\mbox{}
\clearpage{\pagestyle{empty}\cleardoublepage}       %本有

\mainmatter\defaultfont\sloppy\raggedbottom
\makeatletter
\fancypagestyle{plain}{                              % 设置开章页眉页脚风格
    \fancyhf{}
    \fancyhead[RO]{\song\xiaosi \leftmark}
    \fancyhead[LE]{\song\xiaosi \@ctitle}
    \lhead{\song\wuhao {XXXXXX}}  % 左页眉
    \rhead{\song\wuhao \leftmark}    % 右页眉
    \fancyfoot[C]{\song\xiaowu -\,\thepage\,-}           % 首页页脚格式
    \renewcommand{\headrulewidth}{0.5pt}
    \renewcommand{\footrulewidth}{0pt}
}
\makeatother
\setcounter{page}{1}                                 % 单独从 1 开始编页码
%\titleformat{\chapter}{\centering\xiaosan\hei}{\chaptername}{2em}{}

                                       % 恢复chapter 标题格式要求
% !Mode:: "TeX:UTF-8"

\chapter{绪论\label{introduction}}
\clearpage{\pagestyle{empty}\cleardoublepage}       %本有
% !Mode:: "TeX:UTF-8"

\chapter{关键技术原理介绍\label{keytechnology}}
\clearpage{\pagestyle{empty}\cleardoublepage}       %本有
% !Mode:: "TeX:UTF-8"

\chapter{XXXX\label{platform}}
\clearpage{\pagestyle{empty}\cleardoublepage}       %本有
% !Mode:: "TeX:UTF-8"
\chapter{XXXX\label{cooperative_algorithm}}
\clearpage{\pagestyle{empty}\cleardoublepage}       %本有
% !Mode:: "TeX:UTF-8"

\chapter{XXXXX\label{experiment}}
\clearpage{\pagestyle{empty}\cleardoublepage}       %本有
% !Mode:: "TeX:UTF-8"

\chapter{总结与展望\label{conclusion}}

\clearpage{\pagestyle{empty}\cleardoublepage}       %本有

%%%%%%%%%%  参考文献  %%%%%%%%%%
\titleformat{\chapter}{\centering\sihao\hei}{\chaptername}{2em}{}
\defaultfont
%\bibliographystyle{TJUThesis}
\bibliographystyle{references/TJUThesis}        % bst文件名,注意不要后缀
\phantomsection
\markboth{参考文献}{参考文献}
\addcontentsline{toc}{chapter}{参考文献}        % 参考文献加入到中文目录
%\nocite{*}                                     % 若将此命令屏蔽掉,则未引用的文献不会出现在文后的参考文献中
\bibliography{references/reference}             % bib文件名
\include{references/reference}
\clearpage{\pagestyle{empty}\cleardoublepage}   %本有
% !Mode:: "TeX:UTF-8"



\markboth{附录A实验数据}{附录A~~实验数据}

\addcontentsline{toc}{chapter}{附录A~~实验数据} % 添加到目录中
\chapter*{附录A~~实验数据}
本文所使用的实验数据均已公开。
\section*{实验数据获取地址}
\par

\renewcommand{\tablename}{附表}                                     % 插表题头
%\renewcommand{\thetable}{\arabic{chapter}-\arabic{table}}         % 使表编号为 7-1 的格式
\renewcommand{\thetable}{A-\arabic{table}}         % 使表编号为 A-1 的格式
               %附录 实验结果表
\clearpage{\pagestyle{empty}\cleardoublepage}   %本有
% !Mode:: "TeX:UTF-8"

\titlecontents{chapter}[2em]{\vspace{.5\baselineskip}\xiaosan\song}%
             {\prechaptername\CJKnumber{\thecontentslabel}\postchaptername\qquad}{} %
             {}             % 设置该选项为空是为了不让目录中显示页码
\markboth{攻读硕士学位期间科研成果}{攻读硕士学位期间科研成果}
\addcontentsline{toc}{chapter}{攻读硕士学位期间科研成果}
%\setcounter{page}{1}       % 如果需要从该页开始从 1 开始编页,则取消该注释
\chapter*{攻读硕士学位期间科研成果}

\noindent
\textbf{学术论文:}
\begin{enumerate}
\item

\end{enumerate}

\noindent
\textbf{发明专利:}
\begin{enumerate}
\item

\end{enumerate}


\thispagestyle{empty}
                       % 攻读硕士学位期间发表学术论文情况
\clearpage{\pagestyle{empty}\cleardoublepage}   %本有
% !Mode:: "TeX:UTF-8"

%\titlecontents{chapter}[2em]{\vspace{.5\baselineskip}\xiaosan\song}
%             {\prechaptername\CJKnumber{\thecontentslabel}\postchaptername\qquad}{}
%             {}                            % 设置该选项为空是为了不让目录中显示页码
%\fancypagestyle{plain}   % 设置页眉页脚风格,按照教务处规定,此处出现页眉,但是没有页脚(页码)。
%\lhead{}

%\lhead{\song\wuhao 中山大学硕士学位论文}  % 左页眉
%\rhead{\song\wuhao \leftmark}
%\chead{\song\wuhao 中山大学硕士学位论文} % 设置页眉内容
%\lfoot{}
%\cfoot{\song\xiaowu~\thepage~}
%\rfoot{}
\titlecontents{chapter}[2em]{\vspace{.5\baselineskip}\xiaosan\song}%
{\prechaptername\CJKnumber{\thecontentslabel}\postchaptername\qquad}{} %
{}             % 设置该选项为空是为了不让目录中显示页码
\markboth{致\quad 谢}{致\quad 谢}
\addcontentsline{toc}{chapter}{致\quad 谢} % 添加到目录中
\chapter*{致\quad 谢}

\thispagestyle{empty}             % 致谢
\clearpage{\pagestyle{empty}\cleardoublepage}
\clearpage
\end{CJK*}                                      % 结束中文字体使用
\end{document}                                  % 结束全文
